%!TEX root = ../main/main.tex

\section{INTRODUÇÃO}

\subsection{Motivação}
%----
%Giving  everyone  a  fair  chance  to  obtain  a  quality  education  is  a  fundamental  part  of  the  social contract. To improve social mobility and socio-economic outcomes, it is critically important to remove inequalities  in  education  opportunities  and  to  promote  inclusive  growth by broadening the pool  of candidates for high-skilled jobs. \cite[p.~45]{OCDE_2017}

É notável que educação se tornou um tema recorrente nos meios de comunicação, sobretudo recentemente com o aparente aumento de demanda por qualificação. Essa crescente demanda por educação é impulsionada pelos constantes avanços de base tecnológica. De fato, este progresso vem acompanhado  por um lado da ameaça à mão de obra tradicional e por outro, cada vez mais, das exigências por profissionais mais capacitados. Neste sentido, para aumentar as condições de acesso aos cargos do futuro, é essencial que seja dada a devida atenção para as desigualdades no acesso ao aprendizado \cite[p.~45]{OCDE_2017}.
%----

%----
%Higher  levels  of  educational  attainment  are  associated  with  several  positive  economic  and  social outcomes  for  individuals  (see  Indicators  A5,  A6,  A7  and  A8).  Highly  educated  individuals  generally  have better health, are more socially engaged, and have higher employment rates and higher relative earnings. Higher proficiency in literacy and numeracy is also strongly associated with higher levels of formal education (OECD, 2016) \cite[p.~45]{OCDE_2017}

%Individuals  thus  have  incentives  to  pursue  more  education,  and  governments  have  incentives  to  provide appropriate infrastructure and rganisation to support the expansion of higher educational attainment across the population. Over past decades, almost all OECD countries have seen significant increases in educational attainment, especially among the young and among women.\cite[p.~45]{OCDE_2017}

%\textit{Quais são os benefícios para a vida de quem faz ensino superior?}

Educação não gera apenas intelecto, bons empregos e remuneração associada. Sabe se que quanto mais o indivíduo se qualifica, maior é o seu engajamento social e melhor é a sua saúde. É fundamental, portanto que os governos tenham conhecimento desta associação e promovam ações que possibilitem acesso facilitado às fontes de qualificação que deem suporte a essa expansão \cite[p.~45]{OCDE_2017}.
%----

%----
%\textit{O que melhora em relação as oportunidades de inserção no mercado?}

Como importantes fontes de mão de obra para o mercado de trabalho, as universidades tem importância significativa para o desenvolvimento de uma nação. É por meio delas que são forjadas as competências profissionais e ocorre a transferência de tecnologia e o surgimento de novas. Além das competências específicas para atuação no mercado de trabalho, as universidades desempenham papel fundamental na formações dos valores individuais. Nesse sentido, a  prosperidade de uma sociedade e o aprimoramento do ensino superior são temas indissociáveis.
%----

%----
%A área de educação foi bastante privilegiada em termos de alocação de recursos federais na última década. [\ldots] todos os demais itens ali no grupo de despesas de educação] retratados tiveram forte expansão e passaram a consumir parcelas crescentes dos recursos orçamentários disponíveis. A educação desponta como o item de despesa que mais cresceu. Em 2004 os desembolsos para o setor equivaliam a 4\% da receita líquida do Tesouro, tendo passado a 9,3\% em 2014. Um salto nada desprezível de 130\% \cite[p.~1]{Mendes}.

Percebe-se que, especificamente no caso brasileiro, o tema educação vem recebendo cada vez mais atenção governamental em termos de alocação de recursos. De acordo com \citeonline[p.~1]{Mendes}, esta área passou a absorver parcelas significativas de destinações orçamentárias, sobretudo na última década. De acordo com o autor, estas receitas passaram de 4\% em 2004 para 9,3\% em 2014, um acréscimo de 130\% no período. Isso revela uma importante transição da política educacional no país.
%----

%----
%[\ldots] the Federal Government influenced the supply of vacancies in two ways. The first concerns the Federal Government’s policy for the sector, which was apparently based on the supply of a larger number of vacancies through expansion of private organizations (Pinto, 2004). The second one is related to the ``University Education for All'' (ProUni), a program devised by the Brazilian Ministry of Education in 2004 \cite[p.~2]{Zoghbi_Rocha_Mattos_2013}.

%\textit{O que promoveu a expansão do ensino superior no Brasil?}

Entende-se que o expressivo aumento na destinação de recursos está associado à influência do governo federal na criação de vagas. De acordo com \citeonline[p.~2]{Zoghbi_Rocha_Mattos_2013} isso ocorreu por intermédio da expansão de organizações de ensino privadas e pela criação do programa Universidade para Todos (ProUni) em 2004 pelo Ministério da Educação (MEC).
%----

%----
%O desempenho das Instituições de Ensino Superior (IES) tem sido objeto de estudo de diversos pesquisadores da área da economia da educação. Alguns desses estudos, além da análise qualitativa, se utilizam de metodologias quantitativas para mensurar os indicadores de desempenho das IFES. Indicadores como, por exemplo, ``a proporção de alunos por professor e o custo por estudante'' são formas de avaliação do desempenho das universidades \cite[p.~417]{Costa_Souza_Ramos_Silva_2012}.


%\textit{Necessidades de mensuração da qualidade dos cursos superiores}

Com a expansão da oferta, foram aprimorados mecanismos de mensuração da qualidade dos cursos prestados por instituições de ensino superior (IES) publicas e privadas. Com isso, é natural que a análise do desempenho e da qualidade do ensino superior tenha se tornado uma importante área de pesquisas com a utilização diversas metodologias quantitativas e qualitativas. Isso decorre da necessidade de mensurar a eficiência do gasto público em educação, principalmente por intermédio de indicadores, como a proporção de alunos por professor e o custo por estudante, por exemplo \cite[p.~417]{Costa_Souza_Ramos_Silva_2012}.
%----

%----
%Mancebón e Muñiz [Pérez, M. A. M., \& Torrubia, M. J. M. (2003). Aspectos clave de la evaluación de la eficiencia productiva en la educación secundaria. \textit{Papeles de Economía Española}, (95), 162-187.] destacam algumas características inerentes ao setor de produção educacional. São elas: i) a natureza múltipla e intangível do produto – os produtos educacionais podem ser classificados como: conhecimento e habilidades, valores, atitudes, entre outras características; ii) a participação do cliente no processo produtivo – o cliente (aluno) não é meramente um demandante da mercadoria, mas atua de forma decisiva no processo produtivo; iii) a heterogeneidade dos serviços – devido à participação do estudante no processo produtivo, as unidades produtivas se diferenciam umas das outras; iv) a dimensão temporal – os resultados obtidos no processo produtivo podem não ser suficientes para uma mensuração completa da produção do setor educativo, visto que é necessário observar uma trajetória completa da vida dos estudantes; v) o caráter acumulativo do ensino; vi) a incidência de fatores exógenos – essa característica tem como embasamento a denominada educação informal, que não é obtida pelos anos de estudos, mas sim por experiências fora do setor educacional  \cite[417]{Costa_Souza_Ramos_Silva_2012}.

%\textit{O recente aumento da oferta de ensino superior veio acompanhado pela qualidade?}

O grande obstáculo em mensurar a qualidade de cursos ou instituições de ensino superior reside na dificuldade em determinar as características e dimensões importantes para gerar resultados (ou produtos). Neste raciocínio, \citeonline[417]{Costa_Souza_Ramos_Silva_2012} classificam características inerentes ao setor de produção educacional. São elas: 
\begin{enumerate}[label=\roman*)]
	\item a natureza múltipla e intangível do produto: produtos podem ser conhecimento e habilidades, valores, atitudes, entre outros;
	\item a participação do cliente no processo produtivo: o aluno tem participação ativa no processo produtivo e não apenas demanda a ``mercadoria'';
	\item a heterogeneidade dos serviços: uma vez que o estudante atua no processo produtivo, as unidades produtivas se diferenciam umas das outras;
	\item a dimensão temporal: é necessário observar a trajetória completa da vida dos estudantes para mensurar a eficiência do processo produtivo;
	\item o caráter acumulativo do ensino;
	\item a incidência de fatores exógenos: como experiências adquiridas fora do setor educacional.
\end{enumerate}
%----

\subsection{Justificativa}
%----
%\textit{Dificuldade em avaliar.}

A complexidade na avaliação da qualidade dos cursos superiores reside na diversidade de dimensões subjetivas dos participantes, inerente ao processo de educação. Apesar disso, a avaliação é um componente fundamental pois é capaz de determinar, por exemplo, a sensibilidade dos investimentos em relação aos retornos para a sociedade. Com esse intuito, a formulação de indicadores é central para a tomada de decisão em diversos níveis. Com base nos indicadores, políticas de alocação de recursos podem ser desenvolvidas. Da mesma forma, gestores com atuação direta em instituições de ensino podem fundamentar suas decisões com base em dados concretos.
%----

%----

%Os objetivos da avaliação dos cursos de graduação são: a) identificar as condições de ensino oferecidas aos estudantes, em especial as relativas à organização didático pedagógica, corpo social e instalações físicas; b) verificar a articulação entre PDI, Projeto Pedagógico de Curso - PPC, currículo, vocação institucional e inserção regional; c) analisar a aderência às Diretrizes Curriculares Nacionais – DCN’s. \cite[p.~842]{Brito_Regina_2008}

%\textit{O enade é capaz de avaliar a qualidade dos cursos?}

No Brasil, um dos principais instrumentos de avaliação dos cursos superiores é o Exame Nacional de Desempenhos dos Estudantes de ensino Superior (Enade). Este exame é um dos componentes do Sistema Nacional de Avaliação da Educação Superior SINAES estabelecido na Lei 1086/04 de 14 de Abril de 2004 cujas principais diretrizes, de acordo com \citeonline[p.~842]{Brito_Regina_2008}, são:
\begin{enumerate}[label=\roman*)]
\item Avaliação Institucional: auto-avaliação (pelas CPAs\footnote{Comissão Própria de Avaliação. Trata-se de uma comissão responsável pela processo de avaliação interno da instituição. Prestam as informações diretamente ao Inep. Visa garantir a participação da sociedade e da comunidade universitária. \cite[p.~26]{INEP_2018}} e avaliação externa in loco, desenvolvida pelos avaliadores institucionais capacitados pelo INEP nos moldes do SINAES.
\item Avaliação de Curso: pelos pares na avaliação in loco, pelos estudantes, através do ADES (questionário de Avaliação Discente da Educação Superior que é enviado aos estudantes da amostra do ENADE), pelos coordenadores de curso, mediante questionário dos coordenadores e avaliações realizadas pelos professores dos cursos e a CPA.
\item Avaliação do Desempenho dos estudantes ingressantes e concluintes: através de um exame em larga escala aplicado aos estudantes que preenchem os critérios estabelecidos pela legislação vigente.
\end{enumerate}
Ainda de acordo com a autora, o SINAES se estabelece como elemento norteador das políticas de educação superior no Brasil cujo principal foco é avaliar a situação dos cursos de graduação. A avaliação pelo SINAES tem como intuito: 
\begin{enumerate}[label=\roman*)]
\item identificar as condições de ensino oferecidas aos estudantes, em especial as relativas à organização didático pedagógica, corpo social e instalações físicas;
\item verificar a articulação entre PDI (Plano de Desenvolvimento Institucional), Projeto Pedagógico de Curso - PPC, currículo, vocação institucional e inserção regional;
\item analisar a aderência às Diretrizes Curriculares Nacionais – DCN’s.
\end{enumerate}

%----

%----
%\textit{O que avalia o enade?}

%Infelizmente, essas comparações e rankings são de grande gosto popular e propagandístico. A mídia pode elaborar ranqueamentos, pois os resultados são públicos, embora nada contribuam para que as IES melhorem, embora exista a crença disseminada de que ocorrem mudanças quando da divulgação. Os documentos oficiais que tratam do ENADE indicam que, na passagem pela IES, além de dominar os conhecimentos e desenvolver as habilidades e competências necessárias para o perfil da profissão escolhida, espera-se que os graduandos evidenciem a compreensão de temas que transcendam ao seu ambiente próprio de formação e sejam importantes para a realidade contemporânea \cite[847]{Brito_Regina_2008}.

%\textit{Grande marketing na divulgação das notas no Enade.}

A prática de ranqueamento de instituições com base em pontuação no Enade parece ser uma prática comum, principalmente nos mecanismos de marketing de instituições de ensino particulares. Percebe-se que isso é algo que se encontra presente desde que o INEP iniciou os ciclos de avaliação \cite[p.~294]{Verhine_Dantas_Soares_2006}. De fato, existe um grande gosto popular por tais comparações com base em disputa por posições, embora de nada sirvam para mensurar a capacidade de agregação de habilidades e competências dos estudantes nos cursos. \cite[p.~847]{Brito_Regina_2008}.
%----

%----
%\textit{O enade é indutor de qualidade?}

A amplitude de variáveis analisadas pelo Enade, à primeira vista, parece ser capaz de aferir com precisão aspectos evolutivos de componentes individuais dos cursos de graduação. Contudo, uma interpretação mais atenta das diretrizes estabelecidas pelo SINAES leva a entender que evoluções em itens específicos estão aquém dos seus reais objetivos. Há que se investigar a relação entre a diversidade destas variáveis com o vetor qualidade. Para isso é necessário que se racionalize o universo de variáveis em um conjunto delimitado de dimensões.

Diante do exposto, este trabalho é resultado da tentativa de analise das variáveis que constituem o Enade por meio da redução das dimensionalidades. Acredita-se que seja possível extrair perfis de variáveis representativos dos cursos superiores capazes de descrevê-los em arranjos significativos de fácil compreensão.

O propósito deste trabalho consiste em identificar quais variáveis são determinantes na avaliação dos cursos superiores pelo Enade e com isso desenvolver uma melhor compreensão sobre as avaliações do INEP e dos cursos. Procurou-se alcançar esta finalidade por meio da aplicação de técnicas estatísticas de descrição de dados (Análise Exploratória de Dados), determinação de componentes (Análise dos Componentes Principais) e fatores
(Análise de Fator).
%----

\subsection{Estrutura do Trabalho}
%Esta pesquisa encontra-se organizada em seções. A primeira é esta e contém a introdução. Na segunda (Avaliação do Ensino Superior no Brasil) será feito um panorama histórico dos principais instrumentos de avaliação do nível superior no Brasil. A terceira seção (Avaliação do ensino superior no mundo) é destinada a apresentar as práticas internacionais de avaliação bem como a situação dos outros países perante o Brasil. A seção 4 (Mensuração da eficiência de um curso superior) apresenta o estado da arte no que se refere aos principais métodos de análise de indicadores de qualidade de cursos superiores. A quinta seção é destinada ao desenvolvimento da Metodologia. Por fim, as seções 6 e 7 apresentam, respectivamente, os resultados e discussão provenientes desta pesquisa.

%Modificação: remoção da sessão mensuração
Esta pesquisa encontra-se organizada em seções. A presente seção apresenta a introdução. Na segunda (Avaliação do Ensino Superior no Brasil), será feito um panorama histórico dos principais instrumentos de avaliação do nível superior no Brasil. A terceira seção (Avaliação do ensino superior no mundo) é destinada a apresentar as práticas internacionais de avaliação bem como a situação dos outros países perante o Brasil. A quarta seção é destinada ao desenvolvimento da Metodologia. Por fim, as seções 5 e 6 apresentam, respectivamente, os resultados e conclusão provenientes desta pesquisa.

\pagebreak