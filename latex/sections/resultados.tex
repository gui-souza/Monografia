%!TEX root = ../main/main.tex

\section{RESULTADOS}

%Parágrafo incluso, após comentário no Schoology em 10/7 às 9:39 am.
Por meio da aplicação da análise do componente principal foi possível identificar características relevantes sobre as variáveis preditoras do CPC. A referida análise também possibilitou o agrupamento destas variáveis em componentes principais. Verificou-se que o primeiro (principal) componente tem afinidade com a estrutura pedagógica do curso (Organização Didático-pedagógica), sua estrutura (Infra-estrutura e Instalações) e perspectiva de futuro do ponto de vista do aluno (Oportunidades de Ampliação da Formação). O desempenho nestes quesitos tem origem em avaliação subjetiva do aluno participante do censo. A classificação destas variáveis como constantes no primeiro componente permite afirmar que um maior desempenho nestes quesitos implicará em maior resposta na variável dependente (CPC Contínuo).

%Parágrafo incluso, após comentário no Schoology em 10/7 às 9:39 am.
Foi possível verificar notável semelhança nos resultados das análises multivariadas AF e ACP no que se refere a composição do conjunto de variáveis que representam a redução de dimensionalidade. Nesta pesquisa, constatou-se o caráter corroborativo da AF em relação ao ACP pelo fato de os perfis e fatores terem apresentado magnitudes significativas em variáveis semelhantes.

A ACP e AF permitiram identificar as variáveis com maior participação na formação do CPC. A aplicação dessas duas metodologias possibilitou verificar que as variáveis relativas às percepções dos alunos são as que mais influenciam na formação do CPC. As variáveis \textit{Organização Didático-pedagógica, Infra-estrutura e Instalações e Oportunidades de Ampliação da Formação} constituem o componente principal na ACP (pelo acúmulo de 1/4 do espectro de variabilidade) e o fator na AF (pelo acúmulo de 22,7\% da variabilidade). 

%Em relação à primeira variável, é possível afirmar que fatores como coerência da estrutura curricular com os objetivos dos cursos; adequação e atualização das ementas e das disciplinas; uma seleção de conteúdos satisfatória; adequação, atualização e relevância da bibliografia; dentre outros, são favoráveis à melhora do indicador de qualidade do curso de graduação.

%No que se refere à Infra-estrutura e Instalações podem ser citados elementos como: ambientes projetados para atender a todos os requisitos necessários para a realização das atividades de ensino; que levem em consideração o conforto não apenas dos discentes, mas de toda comunidade acadêmica e que sejam adequados para questões relacionadas à acessibilidade. Aliado a estes, sistema de segurança, iluminação, ventilação, equipamentos e mobiliários adequados, podem impactar de forma significativa o CPC.

%Finalmente, as Oportunidades de Ampliação da Formação tem grande importância na composição do CPC. Devem ser criados meios para que, durante o período da graduação, sejam apresentadas aos alunos as alternativas de prosseguimento dos estudos. Arrisca-se afirmar que isso proporciona resultados não apenas no que se refere ao indicador de qualidade do curso superior, mas também implica em avanços em termos de ampliação da fronteira do conhecimento.

No segundo componente da ACP, em relação aos alunos, tem maiores pesos as variáveis \textit{Formação Geral} e \textit{Conhecimentos Específicos}. Já em relação aos professores tem destaque as variáveis \textit{Quantidade de Mestres} e \textit{Quantidade de Doutores} nos cursos. Este componente contabiliza 21,6\% da variabilidade dos dados da ACP e 20,2\% na AF.

O terceiro componente (ou fator) é composto, tanto na ACP quanto na AF, pelas variáveis \textit{Concluíntes Inscritos}, \textit{Concluíntes Participantes} e \textit{Concluíntes Participantes com Nota no Enem}. Na ACP, o componente explica 13,9\% da variabilidade e 12,1\% AF.

O quarto fator (ou componente) é constituído na AF e na ACP pelas variáveis \textit{Concluíntes Inscritos}, \textit{Concluíntes Participantes} e \textit{Concluíntes Perticipantes com Nota no Enem} e nota no \textit{Indicador de Diferença de Desempenho}. Na AF o fator correspondente explica 10,3\% da variabilidade. Já na ACP, 12,3\%.

\pagebreak